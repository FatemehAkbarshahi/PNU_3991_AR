\documentclass{beamer}
%\usetheme[useblacktitletext]{diepen}
\usepackage{graphicx}

% Copyright (C) 2018-2020 Pasquale Claudio Africa and the LaTeX community.
% A full list of contributors can be found at
%
%     https://github.com/elauksap/focus-beamertheme
% 
% This file is part of beamerthemefocus.
% 
% beamerthemefocus is free software: you can redistribute it and/or modify
% it under the terms of the GNU General Public License as published by
% the Free Software Foundation, either version 3 of the License, or
% (at your option) any later version.
% 
% beamerthemefocus is distributed in the hope that it will be useful,
% but WITHOUT ANY WARRANTY; without even the implied warranty of
% MERCHANTABILITY or FITNESS FOR A PARTICULAR PURPOSE. See the
% GNU General Public License for more details.
% 
% You should have received a copy of the GNU General Public License
% along with beamerthemefocus. If not, see <http://www.gnu.org/licenses/>.

\mode<presentation>


% DEFINE COLORS. ---------------------------------------------------------------
\definecolor{main}{RGB}{64, 64, 64}
\definecolor{background}{RGB}{239, 239, 239}

\definecolor{alert}{RGB}{180, 0, 0}
\definecolor{example}{RGB}{0, 110, 0}


% SET COLORS. ------------------------------------------------------------------
\setbeamercolor{normal text}{fg=main, bg=background}
\setbeamercolor{alerted text}{fg=alert}
\setbeamercolor{example text}{fg=example}

\setbeamercolor{titlelike}{fg=background, bg=main}
\setbeamercolor{frametitle}{parent={titlelike}}

\setbeamercolor{footline}{fg=background, bg=main}

\setbeamercolor{block title}{bg=main!80!background, fg=background}
\setbeamercolor{block body}{bg=main!10!background, fg=main}

\setbeamercolor{block title alerted}{bg=alert, fg=background}
\setbeamercolor{block body alerted}{bg=alert!10!background, fg=main}

\setbeamercolor{block title example}{bg=example, fg=background}
\setbeamercolor{block body example}{bg=example!10!background, fg=main}

\setbeamercolor{itemize item}{fg=main}
\setbeamercolor{itemize subitem}{fg=main}

\setbeamercolor{enumerate item}{fg=main!70!black}
\setbeamercolor{enumerate subitem}{fg=main!70!black}

\setbeamercolor{description item}{fg=main!70!black}
\setbeamercolor{description subitem}{fg=main!70!black}

\setbeamercolor{caption name}{fg=main}

\setbeamercolor{section in toc}{fg=main}
\setbeamercolor{subsection in toc}{fg=main}
\setbeamercolor{section number projected}{bg=main}
\setbeamercolor{subsection number projected}{bg=main}

\setbeamercolor{bibliography item}{fg=main}
\setbeamercolor{bibliography entry author}{fg=main!70!black}
\setbeamercolor{bibliography entry title}{fg=main}
\setbeamercolor{bibliography entry location}{fg=main}
\setbeamercolor{bibliography entry note}{fg=main}

\mode<all>

\title{E-research methods,strategies,and issuse by Terry Anderson, Heather Kanuka}
\author{Fatemeh Akbarshahi}


\begin{document}
  {%
    \setbeamertemplate{headline}{}
    \frame{\titlepage}
  }

  \begin{frame}
\frametitle{WHAT DOES THE e IN -RESEARCH MEAN?}
We often joke that adding the lettere in front of every noun we use is an unfortunate distinction of the early years of this Internet technology era. We struggled with the stigma of trendiness that will mark and date a text referring to e-research.\\
 In fact we fear a visit from the "Society for the Preservation of the Other 25 Letters"when they see the effusive use of the prefix used in this book! However, we think the term captures some of the excitement, breadth, and diversity offered by an ever-increasing and sometimes bewildering set of new Net-based tools and techniques.\\
  Only a few years ago (as in email) meant a tool that was primarily text-based, operated on a relatively insecure communications link,
and provided a wide variation in performance and quality of service.\\
 In education, eapplications focused on the lowest common denominators so that students and faculty
could access contents with even the slowest and most dated of hardware. 
  \end{frame}
    \begin{frame}
Convergence  of audio, video, and multimedia channels to a Net-based platform, which is continuing
to fall in price and rise in power has resulted in an explosion of applications in almost every domain.\\
 This has also resulted in a change of our connotations of the Net or the e word. Generally, the e prefix means that the activity or noun modified takes place on a high-speed, digital network that is available "any time/anywhere." Today that network is
the Internet.    
  \end{frame}
  
    \begin{frame}
\frametitle{WHAT EDUCATIONAL RESEARCH ACTIVITIES DOES e-RESEARCH
ENCOMPASS?}  
The Net now supports a wide variety of communication modes and information pro.
cessing tools. As such, it is becoming easier to define the subset of behaviors that can not
be researched on the Net as opposed to those that can be the subject of research. Not
withstanding the dangers of missing novel ways of using the Net, we list below some of
the most obvious manifestations of e-research.
\begin{itemize}
 \item
  Distribution and retrieval of text-based surveys.
  \item
  Open-ended or structured text-based interviews conducted via email or computer
mediated conferencing.
\item
Focus groups using real-time Net-based video or audio conferencing.
\end{itemize}
  \end{frame}
  
  \begin{frame}
  \begin{itemize}
\item
Analysis of Web logs and other tracking tools for measurement and synthesis of
online activities.
\item
Net-based telephone interviews.
\item
Analysis of text transcripts of learning or social activities.
\item
Analysis of social behavior in virtual reality environments.
\item
Online assessment and/or evaluation of performance or knowledge.
  \end{itemize}
  We expect that the increasing power and ubiquity of the Net coupled with its imagi
native use by researchers will result in continuing expansions and variations of the scope
of research practiced around the globe.


  \end{frame}

  \begin{frame}
  \frametitle{THE SPECIAL TASK OF -RESEARCH}
  The networked world is awash in volumes of data. E-research helps us to convert
this data into information and present and disseminate this information in ways that
allow it to be transformed into knowledge and wisdom by the researchers, their sponsors,
educators, and the general public. \\
The quantity of information produced, coupled with the speed in which it can be accessed, filtered, sorted, and combined creates endless
opportunity. \\
However, this abundance forces e-researchers to be more selective and critical
of the veracity of the data they gather. \\
In addition, it is becoming increas ingly apparent
that we can no longer, if we ever could, gather all relevant data.\\
 Instead we must make
judicious decisions about which type and what quantity of data is most helpful in answering
our research questions.
  \end{frame}
  
\begin{frame}
E-research is more than a ser of new research techniques. \\
The quantum physicist
studying subatomic particles realizes that the very act of viewing these tiniest of parti cles
disturbs and changes the objects. \\
The e-researcher is a component of the Net Eresearchers
provide and create tools for analysis and conceptual understanding of human behavior as
it develops on the networks.\\
 In some cases the e-researcher is the outside evaluator, in
other contexts the practitioner e-researcher is both a participant and researcher of the
environment in which the research occurs. \\
E-researchers are also usually members of
other Net communities, thus they bring their experience and insights into the way online
individuals and groups communicate and operate.\\
 They act as Net-savvy artisans of a network culture. \\
 Informally, they interact with peers, family, and coworkers-investing their
time in the development of new skills and in the process gaining "Net efficacy."
  \end{frame}
  
\begin{frame}
E-research takes its place alongside e-commerce and e-learning as alternative ways to
act, understand, and create knowledge in a networked society.\\
 New tools require new
skills, but also allow creativity and an ability to manipulate the world in different ways.\\
These new tools span both the physical and temporal barriers.\\
 We are accustomed to
conceiving of technology spanning geography-after all, humans have had nearly 150 years
since the telegraph first allowed us to communicate in real time over geographie distance.\\
The Net easily meets this challenge. But equally, the Net spans temporal distance. \\
Users
are now able to benefit from asynchronous interaction through the tools of email and
voicemail, or the capture and time shifting of audio or visual presentation. New tools such
as asynchronous voice conferencing and video capture" (an advanced form of picture mail)
promise to allow full multimedia interaction in asynchronous formats.
 \end{frame}

\begin{frame}
Asynchronous communication has also been with us for a long time. \\
From St. Paul’s
letters to the early Christian church to the friendships that have grown and flourished
via pen pal letters - asynchronicity has provided a uniquely reflective means by which
humans communicate and by which we are communicating with you at this very moment.
However, asynchronicity has long been confounded with text literacy.\\
 Now we
realize that text-based communication, supported either asynchronously or in real time
(as practiced in ICQ-an online instant messaging program, MOOS-Mud Object Oriented,
MUDs-Multi-User Dungeons, Palaces, and other Net-based chat systems), is but one form
of communication. In an advanced, Net-based context, voice, sound, and video become
as easily formatted, stored, and retrieved as text. 
 \end{frame}
 
\begin{frame}
Already, early versions of asynchronous
voice conferencing (for example, see www.wimba.com) and asynchronous "virtual people
speaking your email" animations of voice messaging (i.e., http://www.lifefx.com) are
becoming available in addition to synchronous audio and video conferencing.\\
Because the Net so aptly supports both synchronous and asynchronous communication,
it should be no surprise that e-research utilizes this capability to provide a wide variety
of research methods and tool capacities.\\
 Research applications can be cus. tomized
to take advantage of either synchronous or asynchronous formats-or both For example,
online focus groups allow the researcher to gather groups of subjects from widely disbursed
geographic locations. 
 \end{frame}
 
 
\begin{frame}
These groups can be conducted synchro nously using voice or text
formats so that instant feedback is provided to both researchers and participants, and the
immediate presence can be used to build common understandings and ideas. Alternatively
they can be conducted asynchronously, per mitting reflective interactions that are not
dominated by the participants who think and communicate most quickly.\\
E-research also utilizes the distributed data and information processing capacity of the
Net, Stand-alone data processing applications including statistics programs registration
systems, and programs that monitor network activity) are all becoming "Net-enabled" and
thereby can be applied to locations and times that are noncontin gent with the behavior or
process being studied. \\
Thus, e-researchers are able to use research tools, monitor activity,
and collect data without traveling long distances or coordinating local time schedules.
 \end{frame}
 
 
 
\begin{frame}
E-research permits the exploration of new fields of knowledge. As more social and economic
interaction takes place on the networks, new fields of human endeavor are created.\\
Researchers can now study the ways in which students learn online or how online education
and civic groups make decisions and conduct business. These new human activities
grow in economic and political importance daily. \\
These fields of study are not readily
accessible to researchers who cannot access or who lack the skills to proficiently use the
Net. \\
Thus, this text is a guide that can be used for both instrue tion and motivation to
acquire and effectively use the new tools and techniques of networked research.
 \end{frame}
 
 
 
\begin{frame}
 If, as Benedikt (1991) argues, cyberspace "has a geography, a physics, a nature and a
rule, of human law" (p. 123), then obviously it is an environment that can provide insight
into human behavior and nature, through examination of the cultural and sociological
constructs that humans create within this context.\\
 Thus, cyberspace as an evolving and
extremely intricate human context attracts the researcher. \\
It is unclear how many of the
research tools that have been developed, tested, and normed in real communities will be
as useful in virtual contexts.\\
 Likely, existing tools will need to be modified to maximize
their usefulness in this new milieu. Moreover, it is certain that
 \end{frame}
 
 
 
\end{document}